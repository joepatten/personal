%------------------------------------------------
% FILENAME: assignment_0_key.tex
%  PROJECT: mathbootcamp
%   AUTHOR: Brett R. Devine
%    EMAIL: brett.devine@wsu.edu
%  WEBSITE: http://brettdevine.github.io
%------------------------------------------------
\documentclass[a4paper, 11pt]{article}
\usepackage{assignment_0_style}

\title{ Assignment 0 Key }
\author{ Brett Devine }
\date{ Jun 19, 2016 }

\begin{document}
\maketitle

\section{Limits}
\label{sec:limits}

\paragraph{Problem 1} Compute
\begin{align}
    \lim_{x\rightarrow 3} \frac{5x^2 - 8x -13}{x^2-5}  \nonumber
\end{align}

%------------------------------------------------
%   BEGIN Problem 1 Answer
%------------------------------------------------
\paragraph{Problem 1 Answer}
\begin{align}
    \lim_{x\rightarrow 3} \frac{5x^2-8x-13}{x^2-5} &= \frac{5(3)^2-8(3)-13}{(3)^2-5} \nonumber \\
    &= \frac{8}{2} = 2 \nonumber 
\end{align}

%------------------------------------------------
%   END Problem 1 Answer
%------------------------------------------------

\paragraph{Problem 2} Compute
\begin{align}
    \lim_{x\rightarrow 3} \frac{x^4 - 81}{2x^2-5x-3}  \nonumber
\end{align}

%------------------------------------------------
%   BEGIN Problem 2 Answer
%------------------------------------------------

\paragraph{Problem 2 Answer}
Substitution of 3 into the expression will result in the indeterminate form $0/0$ so factor out the expressions causing the indeterminate form
\begin{align}
    \lim_{x\rightarrow 3} \frac{x^4 - 81}{2x^2-5x-3} &= \lim_{x\rightarrow 3} \frac{(x^2-9)(x^2+9)}{(x-3)(2x+1)} \nonumber \\
    &= \lim_{x\rightarrow 3} \frac{(x-3)(x+3)(x^2+9)}{(x-3)(2x+1)} \nonumber \\
    &= \lim_{x\rightarrow 3} \frac{(x+3)(x^2+9)}{(2x+1)} \nonumber \\
    &= \frac{((3) + 3)((3)^2 + 9)}{2(3) + 1} \nonumber \\
    &= \frac{108}{7} \nonumber   
\end{align}

%------------------------------------------------
%   END Problem 2 Answer
%------------------------------------------------

\paragraph{Problem 3}
Compute
\begin{align}
    \lim_{x\rightarrow 4} \frac{3-\sqrt{x+5}}{x-4}  \nonumber
\end{align}

%------------------------------------------------
%   BEGIN Problem 3 Answer
%------------------------------------------------

\paragraph{Problem 3 Answer}
\begin{align}
    &= \lim_{x\rightarrow 4} \frac{3-\sqrt{x+5}}{x-4} \cdot \frac{3+\sqrt{x+5}}{3 + \sqrt{x+5}}  \nonumber \\
    &= \lim_{x\rightarrow 4} \frac{9-(x+5)}{(x-4)(3+\sqrt{x+5})} \nonumber \\
    &= \lim_{x\rightarrow 4} \frac{4-x}{(x-4)(3 + \sqrt{x + 5}} \nonumber \\
    &= \lim_{x\rightarrow 4} \frac{-(x-4)}{(x-4)(3 + \sqrt{x + 5})} \nonumber \\
    &= \lim_{x\rightarrow 4} \frac{-1}{3 + \sqrt{x+5}} \nonumber \\
    &= \frac{-1}{3 + \sqrt{4 + 5}} \nonumber \\
    &= -\frac{1}{6} \nonumber 
\end{align}

%------------------------------------------------
%   END Problem 3 Answer
%------------------------------------------------

\paragraph{Problem 4}
Consider the values of constants $a$ and $b$ so that $\lim_{x \rightarrow 2} f(x)$ exists and is equal to $f(2)$ where $f(x)$ is defined as below.
\begin{align}
    f(x) = \begin{cases}
    			a + bx & \text{ if } x > 2 \\
    			3 & \text{ if } x=2 \\
    			b-ax^2 & \text{ if } x < 2
    	   \end{cases}  \nonumber
\end{align}

%------------------------------------------------
%   BEGIN Problem 4 Answer
%------------------------------------------------

\paragraph{Problem 4 Answer}
\begin{align}
    \lim_{x\rightarrow 2+} f(x) = \lim_{x\rightarrow 2+} (a+bx) &= a + 2b = 3  \nonumber \\
    \lim_{x\rightarrow 2-} f(x) = \lim_{x\rightarrow 2-} (b-ax^2) &= b - 4a = 3 \nonumber
\end{align}
The above expressions give us a system of equations
\begin{align}
    a + 2b &= 3  \nonumber \\
    b-4a &= 3 \nonumber 
\end{align}
Solving the system of 2 equations in 2 unknowns gives us $a = -\frac{1}{3}$ and $b = \frac{5}{3}$.

%------------------------------------------------
%   END Problem 4 Answer
%------------------------------------------------

\paragraph{Problem 5}
Compute the following limit.
\begin{align}
    \lim_{x\rightarrow \infty} \frac{100}{x^2 + 5}  \nonumber
\end{align}

%------------------------------------------------
%   BEGIN Problem 5 Answer
%------------------------------------------------

\paragraph{Problem 5 Answer}
\begin{align}
    \lim_{x\rightarrow \infty} \frac{100}{x^2 + 5} = \frac{100}{\infty} = 0  \nonumber
\end{align}

%------------------------------------------------
%   END Problem 5 Answer
%------------------------------------------------

\paragraph{Problem 6}
Compute
\begin{align}
    \lim_{x\rightarrow \infty} (3x^3 - 1000x^2)  \nonumber
\end{align}

%------------------------------------------------
%   BEGIN Problem 6 Answer
%------------------------------------------------

\paragraph{Problem 6 Answer}
Substitution yields an indeterminate form ``$\infty - \infty$'' which we circumvent by factoring.
\begin{align}
    &= \lim_{x\rightarrow 3} x^2(3x - 1000)  \nonumber \\
    &= \infty \cdot \infty \nonumber
\end{align}
which means that the limit does not exist.

%------------------------------------------------
%   END Problem 6 Answer
%------------------------------------------------

\paragraph{Problem 7}
Compute
\begin{align}
    \lim_{x\rightarrow \infty} \frac{7x^2 + x - 100}{2x^2 - 5x}  \nonumber
\end{align}

%------------------------------------------------
%   BEGIN Problem 7 Answer
%------------------------------------------------

\paragraph{Problem 7 Answer}
\begin{align}
    &= \lim_{x\rightarrow \infty} \frac{7x^2 + x - 100}{2x^2 - 5x} \cdot \frac{\frac{1}{x^2}}{\frac{1}{x^2}} \nonumber \\
    &= \lim_{x\rightarrow \infty} \frac{\frac{7x^2}{x^2} + \frac{x}{x^2} - \frac{100}{x^2}}{ \frac{2x^2}{x^2} - \frac{5x}{x^2} } \nonumber \\
    &= \lim_{x\rightarrow \infty} \frac{ 7 + \frac{1}{x} - \frac{100}{x^2} }{ 2 - \frac{5}{x} } \nonumber \\
    &= \frac{7 + 0 - 0}{2 - 0} \nonumber \\
    &= \frac{7}{2} \nonumber 
\end{align}

%------------------------------------------------
%   END Problem 7 Answer
%------------------------------------------------

\paragraph{Problem 8}
Compute
\begin{align}
    \lim_{x\rightarrow \infty} \left( 3^x + 3^{2x} \right)^{\frac{1}{x}}  \nonumber
\end{align}

%------------------------------------------------
%   BEGIN Problem 8 Answer
%------------------------------------------------

\paragraph{Problem 8 Answer}
\begin{align}
    &= \lim_{x\rightarrow \infty} \left( 3^x + \left[3^2\right]^x \right)^{\frac{1}{x}}  \nonumber \\
    &= \lim_{x\rightarrow \infty} \left( 3^x + 9^x \right)^{\frac{1}{x}} \nonumber \\
    &= \lim_{x\rightarrow \infty} \left( 9^x \left[ \frac{3^x}{9^x} + \frac{9^x}{9^x} \right] \right)^{\frac{1}{x}} \nonumber \\
    &= \lim_{x\rightarrow \infty} \left( 9^x \left[ \left(\frac{3}{9}\right)^x + 1 \right] \right)^{\frac{1}{x}} \nonumber \\
    &= \lim_{x\rightarrow \infty} \left( 9^x \left[ \left(\frac{1}{3}\right) + 1 \right] \right)^{\frac{1}{x}} \nonumber \\
    &= \lim_{x\rightarrow \infty} \left(9^x \right)^{\frac{1}{x}} \cdot \left[ \left(\frac{1}{3}\right)^x + 1 \right]^{\frac{1}{x}} \nonumber \\
    &= \lim_{x\rightarrow \infty} 9 \left[ \left(\frac{1}{3}\right)^x + 1 \right]^{\frac{1}{x}} \nonumber \\
    &= (9)[0 + 1]^0 \nonumber \\
    &= 9 \nonumber 
\end{align}

%------------------------------------------------
%   END Problem 8 Answer
%------------------------------------------------

\paragraph{Problem 9}
Compute $\lim_{x\rightarrow 0^+} \; x \cdot \ln x$.

%------------------------------------------------
%   BEGIN Problem 9 Answer
%------------------------------------------------

\paragraph{Problem 9 Answer}
\begin{align}
    &= \lim_{x\rightarrow 0^+} \frac{\ln x}{1/x} \nonumber \\
    &= \frac{\ln 0}{1/0^+} \nonumber
\end{align}
which leads to the indeterminate form $-\infty / \infty$ so use l'Hopital's Rule.
\begin{align}
    &= \lim_{x\rightarrow 0^+} \frac{1/x}{-1/x^2}  \nonumber \\
    &= \lim_{x\rightarrow 0^+} \frac{1}{x} \cdot \frac{x^2}{-1} \nonumber \\
    &= \lim_{x\rightarrow 0^+} (-x) \nonumber \\
    &= 0 \nonumber 
\end{align}
%------------------------------------------------
%   END Problem 9 Answer
%------------------------------------------------

\paragraph{Problem 10}
Compute $\lim_{x \rightarrow 0^+} x \cdot \left( \ln x \right)^2$.

%------------------------------------------------
%   BEGIN Problem 10 Answer
%------------------------------------------------

\paragraph{Problem 10 Answer}
\begin{align}
    &= \lim_{x\rightarrow 0^+} \frac{\left(\ln x\right)^2}{1/x} = \frac{\infty}{\infty}  \nonumber
\end{align}
Use l'Hopital's Rule
\begin{align}
    &= \lim_{x\rightarrow 0^+} \frac{2\ln x \cdot (1/x)}{\frac{-1}{x^2}} = \frac{\infty}{\infty}  \nonumber
\end{align}
So use l'Hopital's Rule again
\begin{align}
    &= \lim_{x\rightarrow 0^+} \frac{2 \cdot 1/x}{\frac{1}{x^2}}  \nonumber \\
    &= \lim_{x\rightarrow 0^+} 2x \nonumber \\
    &= 2 \cdot 0 \nonumber \\
    &= 0 \nonumber 
\end{align}

%------------------------------------------------
%   END Problem 10 Answer
%------------------------------------------------

\paragraph{Problem 11}
Compute $\lim_{x \rightarrow 0 } (1-x)^{1/x}$.

%------------------------------------------------
%   BEGIN Problem 11 Answer
%------------------------------------------------

\paragraph{Problem 11 Answer}
Rewrite problem
\begin{align}
    \lim_{x\rightarrow 0} (1-x)^{1/x} &= \lim_{x\rightarrow 0} e^{\ln(1-x)^{1/x}}  \nonumber \\
    &= \lim_{x\rightarrow 0} e^{(1/x)\cdot \ln(1-x)} \nonumber \\
    &= e^{\lim_{x\rightarrow 0} \frac{\ln(1-x)}{x} } \nonumber \\
    &= e^{(\ln 1)/0} = e^{0/0} \nonumber
\end{align}
Use l'Hopital's rule for the indeterminate form
\begin{align}
    &= e^{\lim_{x\rightarrow 0} \frac{\frac{1}{1-x} \cdot (-1)}{1} }  \nonumber \\
    &= e^{-1/(1-0)} \nonumber \\
    &= e^{-1} \nonumber \\
    &= \frac{1}{e} \nonumber 
\end{align}
%------------------------------------------------
%   END Problem 11 Answer
%------------------------------------------------

\paragraph{Problem 12}
Consider the following functions for $w \geq 0$ and $0 < \sigma < 1$.
\begin{align}
    u(w) &= \frac{w^{1-\sigma}-1}{1-\sigma}  \nonumber \\
    u(w) &= \ln(w) \nonumber \nonumber \\
    u(w) &= \sqrt{w} \nonumber 
\end{align}
For each function above, find the following limits (if they exist) where $u'(\cdot)$ and $u''(\cdot)$ represent the first and second derivatives respectively.
\begin{enumerate}[(i)]
	\item $\lim_{w\rightarrow 0} \; -u''(w)/u'(w)$.
	\item $\lim_{w\rightarrow \infty} \; -u''(w)/u'(w)$.
	\item $\lim_{w\rightarrow 0} \; (-u''(w)\cdot w)/u'(w)$.
	\item $\lim_{w\rightarrow \infty} \; (-u''(w)\cdot w)/u'(w)$.
\end{enumerate}


\paragraph{Problem 13} Differentiate $y = x^x$ (with respect to $x$).

%------------------------------------------------
%   BEGIN Problem 13 Answer
%------------------------------------------------

\paragraph{Problem 13 Answer}
$x^{x} (1 + \ln x)$.

%------------------------------------------------
%   END Problem 13 Answer
%------------------------------------------------

\paragraph{Problem 14} Differentiate $y = x^{e^x}$ (with respect to $x$).

%------------------------------------------------
%   BEGIN Problem 14 Answer
%------------------------------------------------

\paragraph{Problem 14 Answer} $x^{e^x-1} e^{x} (1 + x \ln x)$

%------------------------------------------------
%   END Problem 14 Answer
%------------------------------------------------

\paragraph{Problem 15} Compute the following limits associated with the functions $f(x) = |x|$.
\begin{align}
    &\lim_{h\rightarrow 0} \frac{f(-2 + h) - f(-2)}{h}  \nonumber \\
    &\lim_{h\rightarrow 0} \frac{f(0 + h) - f(0)}{h} \nonumber \\
    &\lim_{h\rightarrow 0} \frac{f(3 + h) + f(3)}{h} \nonumber
\end{align}

\paragraph{Problem 16}
Define the intervals (if any) over which the following function is continuous.
\[
	f(x) = \frac{7x^5 + x - 2}{x^2-4}
\]

%------------------------------------------------
%   BEGIN Problem 16 Answer
%------------------------------------------------

\paragraph{Problem 16 Answer} \; $(-\infty, -2) \cup (-2,2) \cup (2, \infty)$.

%------------------------------------------------
%   END Problem 16 Answer
%------------------------------------------------

\paragraph{Problem 17}
Show that there is a root of the equation $3x^7 - 2x^5 + x -1 = 0$ between $0$ and $1$.

%------------------------------------------------
%   BEGIN Problem 17 Answer
%------------------------------------------------

\paragraph{Problem 17 Answer}
We see that $f(0) = -1 < 0$ and $f(1) = 3 - 2 + 1 - 1 = 1 > 0$.
By the Intermediate Value Theorem it follows that there exists a number $c$ in $(0,1)$ such that $f(c)=0$ since $f(x)$ is continuous and 0 is between $f(0)$ and $f(1)$.

%------------------------------------------------
%   END Problem 17 Answer
%------------------------------------------------

\paragraph{Problem 18}
Compute the following limit associated with $f(x) = x^2 - 8x + 9$.
\[
	\lim_{x\rightarrow a} \; \frac{(x^2-8x+9)-(a^2-8a+9)}{x-a}
\]

%------------------------------------------------
%   BEGIN Problem 18 Answer
%------------------------------------------------

\paragraph{Problem 18 Answer}
\; $2a - 8$.

%------------------------------------------------
%   END Problem 18 Answer
%------------------------------------------------

\paragraph{Problem 19}
If the function $f(x)$ is differentiable in the interval $(a,b)$ and $|f'(x)| \leq B < \infty$ for all $x$ in the interval $(a,b)$ then is the maximum change in the function over any sub interval $(c,d) \subseteq (a,b)$ finite or infinite? Prove it.

\paragraph{Problem 20}
If the function $f(x)$ is differentiable on $(a,b)$, but not continuously differentiable, then is $f$ continuous everywhere on $(a,b)$? Prove it.

\paragraph{Problem 21}
Can you differentiate the expression $2x = 1$? If so what is the derivative?  If not, why not?

\paragraph{Problem 22} Find the derivative of the function
\[
	f(x) = \frac{\sqrt[4]{x}}{x^{-1} \sqrt{x^5}}
\]

%------------------------------------------------
%   BEGIN Problem 22 Answer
%------------------------------------------------

\paragraph{Problem 22 Answer}
\[
	f'(x) = - \frac{5}{4x^2 \sqrt[4]{x} }
\]

%------------------------------------------------
%   END Problem 22 Answer
%------------------------------------------------

\paragraph{Problem 23} Differentiate the function
\[
	f(x) = \frac{3x^2 - 5\sqrt{x} }{ 6x^4 }
\]

%------------------------------------------------
%   BEGIN Problem 23 Answer
%------------------------------------------------

\paragraph{Problem 23 Answer}
\[
	f'(x) = -x^{-3} + \frac{35}{12} x^{-9/2}
\]

%------------------------------------------------
%   END Problem 23 Answer
%------------------------------------------------

\paragraph{Problem 24}
Find the linearization of the function $f(x) = \sqrt[3]{1+x}$ at $a=0$ and use it to approximate the numbers $f(-0.05)$ and $f(0.1)$.
Are these approximations overestimates or underestimates?

%------------------------------------------------
%   BEGIN Problem 24 Answer
%------------------------------------------------

\paragraph{Problem 24 Answer}
\begin{align}
    L(x) &= f(0) + f'(0)(x-0) = 1 + \frac{x}{3}  \nonumber \\
    L(-0.05) &= 0.9833 \approx 0.98305 \quad \text{over estimate} \nonumber \\
    L(0.1) &= 1.0333 \approx 1.0323 \quad \text{over estimate} \nonumber
\end{align}

%------------------------------------------------
%   END Problem 24 Answer
%------------------------------------------------

\paragraph{Problem 25}
Let $f(x) = (3x-5)/(4-2x)$ and find $f^{-1}(x)$.  Then compare $f'(x)$ and $\frac{d}{dx} f^{-1}(x)$ and describe the relationship (if any).

%------------------------------------------------
%   BEGIN Problem 25 Answer
%------------------------------------------------

\paragraph{Problem 25 Answer}
The inverse function is
\[
	f^{-1}(x) = \frac{4x+5}{3+2x}
\]

%------------------------------------------------
%   END Problem 25 Answer
%------------------------------------------------

\paragraph{Problem 26}
Compute the derivative of the following function, $H(p)$, with respect to $p$ and use it to show the answer to the following questions.
\begin{align}
    H(p) &= p \log_2 \left(\frac{1}{p} \right) + (1-p)\log_2 \left(\frac{1}{1-p} \right)  \nonumber
\end{align}
\begin{enumerate}[(i)]
	\item Is there a global maximum and minimum over the interval $[0,1]$, if so, what is it?
	\item If so, over what subset of $[0,1]$, if any, is the function increasing?
	\item If so, over what subset of $[0,1]$, if any, is the function decreasing?
\end{enumerate}

%------------------------------------------------
%   BEGIN Problem 26 Answer
%------------------------------------------------

\paragraph{Problem 26 Answer}
\begin{enumerate}[(i)]
	\item Yes, there is a global max because it is a continuous function on a compact set.  The maximum is $p^* = 0.5$.
	\item The function increases over $[0,0.5]$.
	\item The function decreases over $[0.5,1]$.
\end{enumerate}

%------------------------------------------------
%   END Problem 26 Answer
%------------------------------------------------

\paragraph{Problem 27}
Consider the function $f(x) = x^4 e^x$ with domain all real numbers.
\begin{enumerate}[(i)]
	\item Find the $x$-value(s) of all roots ($x$-intercepts) of $f$.
	\item Find the $x$- and $y$-value(s) of all critical points and identify each as a local max, local min, or neither.
	\item Find the $x$- and $y$-value(s) of all global extrema and identify each as a global max or global min.
	\item Find the $x$-value(s) of all inflection points.
\end{enumerate}

%------------------------------------------------
%   BEGIN Problem 27 Answer
%------------------------------------------------

\paragraph{Problem 27 Answer}
\begin{enumerate}[(i)]
	\item The equation $x^4e^x=0$ means $x^4=0 \implies x=0$ so the only root is $x=0$.
	\item $f'(x) = 4x^3 e^x + x^4 e^x = 0 \implies x = 0,-4$.  Plugging in values shows us that $f$ has a local max at $(-4, 256e^{-4})$ and a local min at $(0,0)$.
	\item There is a global minimum at $(0,0)$, but there is no global maximum since $f(x)\rightarrow \infty$ as $x\rightarrow \infty$.
	\item Look for solutions to $f''(x) = 0$ which yields $x=\{0,-2,-6\}$.  Checking the value of $f$ and $f''$ at those points tells us that the inflection points are $x=\{-2,-6\}$ but not $x=0$.
\end{enumerate}

%------------------------------------------------
%   END Problem 27 Answer
%------------------------------------------------

\paragraph{Problem 28}
Use the Intermediate Value Theorem to show that $f(x) = x^3 - 2x - 1$ has a root on $[1,2]$.

%------------------------------------------------
%   BEGIN Problem 28 Answer
%------------------------------------------------

IVT: if $f$ is continuous on $[a,b]$ and $y$ is a number between $f(a)$ and $f(b)$, then there is a number $c$ between $a$ and $b$ such that $f(c)=y$.
For the function given above, $f(1)=-2$ and $f(2)=3$.
Since $0$ is a number between $-2$ and $3$, the IVT says there is a number $c$ between $1$ and $2$ such that $f(c)=0$; this $c$ is the desired root.

%------------------------------------------------
%   END Problem 28 Answer
%------------------------------------------------

\paragraph{Problem 29}
Does the Extreme Value Theorem say anything about the function $f(x) = x^2$ on each of the following intervals? If so what does it say?  In either case, explain why.
\begin{enumerate}[(i)]
	\item $[1,4]$
	\item $(1,4)$
\end{enumerate}

%------------------------------------------------
%   BEGIN Problem 29 Answer
%------------------------------------------------

\paragraph{Problem 29 Answer}
\begin{enumerate}[(i)]
	\item $f$ has a maximum and a mininum on $[1,4]$ by EVT.
	\item Because $(1,4)$ is not compact, the EVT does not apply and so although there may be, the theorem doesn't say anything to us in this case.
\end{enumerate}

%------------------------------------------------
%   END Problem 29 Answer
%------------------------------------------------

\paragraph{Problem 30}
Find the value of the constant $c$ that the Mean Value Theorem specifies for $f(x)=x^3 + x$ on the interval $[0,3]$.

%------------------------------------------------
%   BEGIN Problem 30 Answer
%------------------------------------------------

\paragraph{Problem 30 Answer}
MVT: If $f$ is continuous on $[a,b]$, then there is a number $c$ between $a$ and $b$ such that $f'(c) = (f(b)-f(a))/(b-a)$.

For the provided function we have $\frac{f(3)-f(0)}{3-0} = \frac{30-0}{3} = 10$.
And $f'(x) = 3x^2 + 1$ so we know $f'(c) = 3c^2 + 1$.
Solving $f'(c)=3c^2 + 1 = 10$ for $c$ yields $c = \sqrt{3}$.

%------------------------------------------------
%   END Problem 30 Answer
%------------------------------------------------

\paragraph{Problem 31}
For the equation $x^3 + y^3 = \ln(xy) - 1$ use implicit differentiation to find $\dd y/ \dd x$.

%------------------------------------------------
%   BEGIN Problem 31 Answer
%------------------------------------------------

\paragraph{Problem 31 Answer}
\begin{align}
    \frac{\dd y}{\dd x} &= \frac{y - 3x^3 y}{3xy^3-x} = \frac{\frac{1}{x}-3x^2}{3y^2-\frac{1}{y}}  \nonumber
\end{align}

%------------------------------------------------
%   END Problem 31 Answer
%------------------------------------------------

\paragraph{Problem 32}
Consider the function
\[
	f(x) = \begin{cases}
				b-x^2 & \text{ if } x < 3 \\
				ax & \text{ if } x \geq 3
	 	   \end{cases}
\]
\begin{enumerate}[(i)]
	\item What condition(s) must be placed on the constants $a$ and $b$ in order for $f$ to be continuous on $(-\infty, \infty)$?
	\item For what values of the constants $a$ and $b$ will $f$ be differentiable on $(-\infty, \infty)$?
\end{enumerate}

%------------------------------------------------
%   BEGIN Problem 32 Answer
%------------------------------------------------

\paragraph{Problem 32 Answer}
\begin{enumerate}[(i)]
	\item The only problem area is $x=3$ so we want $\lim_{x \rightarrow 3^-} f(x) = \lim_{x\rightarrow 3^+} f(x)$.
	This gives us limits $b-9$ and $3a$ so we need $3a=b-9 \iff a = (1/3)b - 3$.
	\item Diffentiate piecewise to find $f'(x)$ for $x < 3$ and $x > 3$ giving us $-2x$ and $a$ respectively.
	Existence will require that $\lim_{x\rightarrow 3^-} f'(x) = \lim_{x\rightarrow 3^+} f'(x) = f'(3)$.
	The piecewise limits of the derivatives are $-6$ and $a$ so $a=-6$ and using the equation found before we know $3(-6) = b-9$ which implies $-18 + 9 = -9 = b$ so we have $\{a=-6, b=-9\}$.
\end{enumerate}

%------------------------------------------------
%   END Problem 32 Answer
%------------------------------------------------

\paragraph{Problem 33}
Consider the function $f(x) = \ln(x^2)$.
Find the fourth-order Taylor polynomial for $f(x)$ centered at $x_0=1$.

%------------------------------------------------
%   BEGIN Problem 33 Answer
%------------------------------------------------

\paragraph{Problem 33 Answer}
\[
	P_3(x) = 2(x-1) - (x-1)^2 + \frac{2}{3}(x-1)^3 - \frac{1}{2}(x-1)^4
\]

%------------------------------------------------
%   END Problem 33 Answer
%------------------------------------------------

\paragraph{Problem 34}
Let $X$ be a random variable with a probability density function (PDF) $f(x)$
\[
	f(x) = \begin{cases}
 				ce^{-x/3} & \text{ for } x > 0 \\
 				0 & \text{ otherwise }
 		   \end{cases}
\]
Remember that a PDF has the property that $\int_{\infty}^{\infty} f(x)\dd x = 1$.
\begin{enumerate}[(i)]
	\item Find the value of the constant $c$ that makes $f(x)$ a valid PDF.
	\item Find the probability that $X \leq 1/4$.
\end{enumerate}

%------------------------------------------------
%   BEGIN Problem 34 Answer
%------------------------------------------------

\paragraph{Problem 34 Answer}
\begin{enumerate}[(i)]
	\item $c = - \frac{1}{3}$.
	\item $1 - e^{-1/12}$.
\end{enumerate}

%------------------------------------------------
%   END Problem 34 Answer
%------------------------------------------------

\paragraph{Problem 35}
Find the derivative of the following function
\[
	G(x) = \int_{1}^{\sin{x}} t \dd t
\]

%------------------------------------------------
%   BEGIN Problem 35 Answer
%------------------------------------------------

\paragraph{Problem 35 Answer}
\[
	G'(x) = \sin{x} \cdot \cos{x}
\]

%------------------------------------------------
%   END Problem 35 Answer
%------------------------------------------------

\paragraph{Problem 36}
Find the derivative of the following function
\[
	G(x) = \int_{72}^{\ln(x)} t \dd t
\]

%------------------------------------------------
%   BEGIN Problem 36 Answer
%------------------------------------------------

\paragraph{Problem 36 Answer}
\[
	G'(x) = \frac{\ln x}{x}
\]

%------------------------------------------------
%   END Problem 36 Answer
%------------------------------------------------

\paragraph{Problem 37}
Suppose that $\sum_{n=1}^{\infty} a_n$ represents a convergent series and that no term of the series equals zero, i.e., $a_n \neq 0$ for all $n=1,2, \dots$.
Prove that $\sum_{n=1}^{\infty} 1/a_n$ is a divergent series.

%------------------------------------------------
%   BEGIN Problem 37 Answer
%------------------------------------------------

\paragraph{Problem 37 Answer}
If $\sum_{n=1}^{\infty} a_n$ is a convergent series, then $a_n \rightarrow 0$ (i.e., the terms go to zero).
If $a_n \rightarrow 0$, then $|1/a_n| \rightarrow \infty $.
As the terms of $\sum_{n=1}^{\infty} 1/a_n$ do not go to zero, by the $n$th term test for divergence, the series diverges.

%------------------------------------------------
%   END Problem 37 Answer
%------------------------------------------------

\paragraph{Problem 38}
Determine whether the give sequence is increasing, decreasing, or not monotonic.
Is the sequence bounded?  On the basis of what you find, does the series converge, diverge or can't be determined?
\[
	a_n = \frac{1}{5^n}
\]

%------------------------------------------------
%   BEGIN Problem 38 Answer
%------------------------------------------------

\paragraph{Problem 38 Answer}
The sequence is \textbf{decreasing} and the sequence is \textbf{bounded} between 0 and 1 and by the monotonic sequence theorem the sequence is \textbf{convergent}.

%------------------------------------------------
%   END Problem 38 Answer
%------------------------------------------------

\paragraph{Problem 39}
Differentiate the function $y = 3(x^2-1)^3 (x^2 + 1)^5$.

%------------------------------------------------
%   BEGIN Problem 39 Answer
%------------------------------------------------

\paragraph{Problem 39 Answer}
\[
	y' = 18x(x^2-1)^2 (x^2+1)^5 + 30x(x^2-1)^3 (x^2 + 1)^4
\]

%------------------------------------------------
%   END Problem 39 Answer
%------------------------------------------------

\paragraph{Problem 40}
Consider a continuously differentiable utility function $u(\cdot)$ such that $u'>0$ and $u'' < 0$.
Utility comes from income $I$ which takes on two different values $I_H(x)$ and $I_L(x)$ and that the probability of that $I=I_H$ is given by $p(x)$ where $x\geq 0$ and $p'(x)>0$.
Differentiate the expected utility with respect to $x$.
\[
	E[u] = p(x)u[I_H(x)] + (1-p(x))u[I_L(x)]
\]


\paragraph{Problem 41}
Consider a twice continuously differentiable utility function $U(\cdot)$ such that $U'>0$ and $U''< 0$.
Utility comes from wealth $W$ and a person takes a gamble $h$ that represents a gain or loss to the a person's wealth and $h=0$ on average ($E(h)=0$).
Let $p$ represent the size of an insurance premium paid to avoid taking the gamble and makes the person exactly indifferent between the gamble $h$ and paying $p$ with certainty.
Let $E$ represent the expectation operator.
Use the identity below to answer the questions.
\[
	E[U(W + h)] \equiv U(W-p)
\]
\begin{enumerate}[(i)]
	\item Derive a linear (first order) Taylor approximation of the right-hand side of the identity.
	\item Derive a quadratic (second order) Taylor approximation of the left-hand side of the identity.
	\item Let $r(W) = -U''(W)/U'(W)$ (this is the Arrow-Pratt measure of risk-aversion) use the previous approximations and the identity relationship to sign the derivative $\dd P / \dd r(W)$.  What is the sign and what relationship does this imply about the level of risk aversion and the amount of insurance premium an individual is willing to pay?
\end{enumerate}


\paragraph{Problem 42}
A new machine as a rental rate $v(s)$ at any time $s$ and depreciates at rate $d$.
With interest rate $r$ the present discounted value of this machine is 
\[
	PDV(t) = \int_{t}^{\infty} e^{(r+d)t} v(s) e^{-(r+d)s} \dd s
\]
Let $p(t)$ be the purchase price of the machine at time $t$.
In equilibrium the purchase price at time $t$, $p(t)$ will equal the present discounted value and hence we have
\[
	p(t) = \int_{t}^{\infty} e^{(r+d)t} v(s)e^{-(r+d)s} \dd s
\]
\begin{enumerate}[(i)]
	\item Using the last expression for $p(t)$, find $\frac{dp(t)}{dt}$.
	\item Using your expression for $\frac{dp(t)}{dt}$ solve for $v(t)$ in terms of $p(t)$ to show the relationship between a new machines rental rate at time $t$ and the machine's purchase price at time $t$.
\end{enumerate}


\paragraph{Problem 43}
A forester must decide when to cut down a growing tree.
The value at any time $t$ is given by $f(t)$ where $f' > 0$ and $f'' < 0$ and there was an initial investment of $L$.
The continuous interest rate is $r$.
The forester must choose $t$ (the time of harvest) to maximize the present discounted value of her profits.
\[
	PDV(t) = e^{-rt}f(t) - L
\]
\begin{enumerate}[(i)]
	\item Differentiate $PDV(t)$ with respect to $t$.
	\item Use the derivative as a first order condition to characterize the relationship between the $t$ that maximizes the forester's present discounted value of profit and the interest rate $r$.
\end{enumerate}























\end{document}